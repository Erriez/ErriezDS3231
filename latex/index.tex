\href{https://travis-ci.org/Erriez/ErriezDS3231}{\tt }

This is an advanced \hyperlink{class_d_s3231}{D\+S3231} high precision I2C R\+TC library for Arduino.



\subsection*{Library features}


\begin{DoxyItemize}
\item Read time
\item Set time
\item Read date and time
\item Set date and time
\item Read Unix Epoch U\+TC 32-\/bit timestamp
\item Read temperature (0.\+25 degree resolution)
\item Alarm 1 (second/minute/hour/day/date match)
\item Alarm 2 (minute/hour/day/date match)
\item Polling and Alarm {\ttfamily I\+N\+T/\+S\+QW} interrupt pin
\item Control {\ttfamily 32k\+Hz} out signal (enable/disable)
\item Control {\ttfamily S\+QW} signal (disable/1/1024/4096/8192\+Hz)
\item Configure aging offset
\item Serial terminal interface
\item Full R\+TC register access
\item Easy debug functionality
\item Basic and advanced examples
\item Set date/time over serial with Python script
\item Low R\+AM footprint\+:
\begin{DoxyItemize}
\item {\ttfamily sizeof(\+D\+S3231)}\+: 1 Byte R\+TC object.
\item {\ttfamily sizeof(\+D\+S3231\+\_\+\+Date\+Time)}\+: 8 Bytes date/time object.
\end{DoxyItemize}
\item Full Doxygen documentation
\end{DoxyItemize}

\subsection*{Hardware}

Any Arduino hardware with a T\+WI interface and {\ttfamily Wire.\+h} support.



\subsection*{Pins}

\tabulinesep=1mm
\begin{longtabu} spread 0pt [c]{*6{|X[-1]}|}
\hline
\rowcolor{\tableheadbgcolor}{\bf Pins board -\/ \hyperlink{class_d_s3231}{D\+S3231} }&\PBS\centering {\bf V\+CC }&\PBS\centering {\bf G\+ND }&\PBS\centering {\bf S\+DA }&\PBS\centering {\bf S\+CL }&\PBS\centering {\bf S\+QW  }\\\cline{1-6}
\endfirsthead
\hline
\endfoot
\hline
\rowcolor{\tableheadbgcolor}{\bf Pins board -\/ \hyperlink{class_d_s3231}{D\+S3231} }&\PBS\centering {\bf V\+CC }&\PBS\centering {\bf G\+ND }&\PBS\centering {\bf S\+DA }&\PBS\centering {\bf S\+CL }&\PBS\centering {\bf S\+QW  }\\\cline{1-6}
\endhead
Arduino U\+NO (A\+T\+Mega328 boards) &\PBS\centering 5V &\PBS\centering G\+ND &\PBS\centering A4 &\PBS\centering A5 &\PBS\centering D2 (I\+N\+T0) \\\cline{1-6}
Arduino Mega2560 &\PBS\centering 5V &\PBS\centering G\+ND &\PBS\centering D20 &\PBS\centering D21 &\PBS\centering D2 (I\+N\+T4) \\\cline{1-6}
Arduino Leonardo &\PBS\centering 5V &\PBS\centering G\+ND &\PBS\centering D2 &\PBS\centering D3 &\PBS\centering D7 (I\+N\+T6) \\\cline{1-6}
Arduino D\+UE (A\+T\+S\+A\+M3\+X8E) &\PBS\centering 3\+V3 &\PBS\centering G\+ND &\PBS\centering 20 &\PBS\centering 21 &\PBS\centering 2 \\\cline{1-6}
E\+S\+P8266 &\PBS\centering 3\+V3 &\PBS\centering G\+ND &\PBS\centering G\+P\+I\+O4 (D2) &\PBS\centering G\+P\+I\+O5 (D1) &\PBS\centering G\+P\+I\+O0 (D3) \\\cline{1-6}
E\+S\+P32 &\PBS\centering 3\+V3 &\PBS\centering G\+ND &\PBS\centering G\+P\+I\+O21 &\PBS\centering G\+P\+I\+O22 &\PBS\centering G\+P\+I\+O0 \\\cline{1-6}
\end{longtabu}
Tested boards\+:


\begin{DoxyItemize}
\item {\bfseries E\+S\+P8266 boards}\+: E\+S\+P12E / We\+Mos D1 \& R2 / Node M\+CU v2 / v3
\item {\bfseries E\+S\+P32 boards\+:} We\+Mos L\+O\+L\+I\+N32 / L\+O\+L\+IN D32
\end{DoxyItemize}

\subsection*{Examples}

Arduino I\+DE $\vert$ Examples $\vert$ Erriez \hyperlink{class_d_s3231}{D\+S3231} R\+TC\+:


\begin{DoxyItemize}
\item \href{https://github.com/Erriez/ErriezDS3231/blob/master/examples/AgingOffset/AgingOffset.ino}{\tt Aging\+Offset} Aging offset programming.
\item \href{https://github.com/Erriez/ErriezDS3231/blob/master/examples/AlarmInterrupt/AlarmInterrupt.ino}{\tt Alarm\+Interrupt} Alarm with interrupts.
\item \href{https://github.com/Erriez/ErriezDS3231/blob/master/examples/AlarmPolling/AlarmPolling.ino}{\tt Alarm\+Polling} Alarm polled.
\item \href{https://github.com/Erriez/ErriezDS3231/blob/master/examples/GettingStarted/GettingStarted.ino}{\tt Getting\+Started} Getting started example which contains most date/time/temperature features.
\item \href{https://github.com/Erriez/ErriezDS3231/blob/master/examples/Minimum/Minimum.ino}{\tt Minimum} Minimum example to read time.
\item \href{https://github.com/Erriez/ErriezDS3231/blob/master/examples/PrintDiagnostics/PrintDiagnostics.ino}{\tt Print\+Diagnostics} Print diagnostics and registers.
\item \href{https://github.com/Erriez/ErriezDS3231/blob/master/examples/ReadTimeInterrupt/ReadTimeInterrupt.ino}{\tt Read\+Time\+Interrupt} Read time with 1\+Hz S\+QW interrupt. (Highly recommended)
\item \href{https://github.com/Erriez/ErriezDS3231/blob/master/examples/ReadTimePolled/ReadTimePolled.ino}{\tt Read\+Time\+Polled} Read time polled.
\item \href{https://github.com/Erriez/ErriezDS3231/blob/master/examples/SetDateTime/SetDateTime.ino}{\tt Set\+Date\+Time} Set date time. (Must be started first)
\item \href{https://github.com/Erriez/ErriezDS3231/blob/master/examples/Temperature/Temperature.ino}{\tt Temperature} Temperature.
\item \href{https://github.com/Erriez/ErriezDS3231/blob/master/examples/Terminal/Terminal.ino}{\tt Terminal} Advanced terminal interface with \href{https://github.com/Erriez/ErriezDS3231/blob/master/examples/Terminal/Terminal.py}{\tt set date/time Python} script.
\end{DoxyItemize}

\subsection*{Documentation}


\begin{DoxyItemize}
\item \href{https://erriez.github.io/ErriezDS3231}{\tt Doxygen online H\+T\+ML}
\item \href{https://github.com/Erriez/ErriezDS3231/raw/gh-pages/latex/ErriezDS3231.pdf}{\tt Doxygen P\+DF}
\item \href{https://github.com/Erriez/ErriezDS3231/blob/master/extras/DS3231.pdf}{\tt D\+S3231 datasheet}
\end{DoxyItemize}

\subsection*{Usage}

{\bfseries Initialization}


\begin{DoxyCode}
1 \{c++\}
2 #include <Wire.h>
3 #include <ErriezDS3231.h>
4 
5 // Create DS3231 RTC object
6 static DS3231 rtc;
7 
8 
9 void setup()
10 \{
11     // Initialize TWI with a 100kHz (default) or 400kHz clock
12     Wire.begin();
13     Wire.setClock(400000);
14 
15     // Initialize RTC
16     while (rtc.begin()) \{
17         // Error: Could not detect DS3231 RTC, retry after some time
18         delay(3000);
19     \}
20 \}
\end{DoxyCode}


{\bfseries Check oscillator status at startup}


\begin{DoxyCode}
1 \{c++\}
2 // Check oscillator status
3 if (rtc.isOscillatorStopped()) \{
4     // Error: DS3231 RTC oscillator stopped. Date/time cannot be trusted. 
5     // Set new date/time before reading date/time.
6     while (1) \{
7         ;
8     \}
9 \}
\end{DoxyCode}


{\bfseries Set time}


\begin{DoxyCode}
1 \{c++\}
2 // Write time to RTC
3 if (rtc.setTime(12, 0, 0)) \{
4     // Error: Write time failed
5 \}
\end{DoxyCode}


{\bfseries Get time}


\begin{DoxyCode}
1 \{c++\}
2 uint8\_t hour;
3 uint8\_t minute;
4 uint8\_t second;
5 
6 // Read time from RTC
7 if (rtc.getTime(&hour, &minute, &second)) \{
8     // Error: Read time failed
9 \}
\end{DoxyCode}


{\bfseries Set date time}


\begin{DoxyCode}
1 \{c++\}
2 // Create and initialize date time object
3 static DS3231\_DateTime dt = \{
4     .second = 0,
5     .minute = 36,
6     .hour = 21,
7     .dayWeek = 7, // 1 = Monday
8     .dayMonth = 29,
9     .month = 7,
10     .year = 2018
11 \};
12 
13 // Set new RTC date/time
14 rtc.setDateTime(&dt);
\end{DoxyCode}


{\bfseries Get date time}


\begin{DoxyCode}
1 \{c++\}
2 DS3231\_DateTime dt;
3 
4 // Read RTC date and time from RTC
5 if (rtc.getDateTime(&dt)) \{
6     // Error: Read date time failed
7 \}
\end{DoxyCode}


{\bfseries Get Epoch Unix U\+TC time}


\begin{DoxyCode}
1 \{c++\}
2 uint32\_t epoch;
3 
4 // Read date/time from RTC
5 if (rtc.getDateTime(&dt)) \{
6     // Error: Read date/time failed
7     return;
8 \}
9 
10 // Convert date/time to 32-bit epoch time
11 epoch = rtc.getEpochTime(&dt));
\end{DoxyCode}


{\bfseries Get temperature}


\begin{DoxyCode}
1 \{c++\}
2 int8\_t temperature;
3 uint8\_t fraction;
4 
5 // Force temperature conversion
6 // Without this call, it takes 64 seconds before the temperature is updated.
7 rtc.startTemperatureConversion();
8 
9 // Read temperature
10 rtc.getTemperature(&temperature, &fraction);
11 
12 // Print temperature. The output below is for example: 28.25C
13 Serial.print(temperature);
14 Serial.print(F("."));
15 Serial.print(fraction);
16 Serial.println(F("C"));
\end{DoxyCode}


{\bfseries Program Alarm 1}

Note\+: Alarm 1 and Alarm 2 have different behavior. Please refer to the documentation which {\ttfamily Alarm1\+Type} and {\ttfamily Alarm2\+Type} are supported. Some examples\+:


\begin{DoxyCode}
1 \{c++\}
2 // Generate alarm 1 every second
3 rtc.setAlarm1(Alarm1EverySecond, 0, 0, 0, 0);
4 
5 // Generate alarm 1 every minute and second match
6 rtc.setAlarm1(Alarm1EverySecond, 0, 0, 45, 30);
7 
8 // Generate alarm 1 every day, hour, minute and second match
9 rtc.setAlarm1(Alarm1MatchDay, 
10               1,  // Alarm day match (1 = Monday)
11               12, // Alarm hour match
12               45, // Alarm minute match
13               30  // Alarm second match
14 );
\end{DoxyCode}


{\bfseries Program Alarm 2}


\begin{DoxyCode}
1 \{c++\}
2 // Generate alarm 2 every minute
3 rtc.setAlarm2(Alarm2EveryMinute, 0, 0, 0);
4 
5 // Generate alarm 2 every hour, minute match
6 rtc.setAlarm2(Alarm2MatchHours, 0, 23, 59);
7 
8 // Generate alarm 2 every date, hour, minute match
9 rtc.setAlarm2(Alarm2MatchDate, 28, 7, 0);
\end{DoxyCode}


{\bfseries Alarm polling}

Note\+: The {\ttfamily I\+NT} pin changes to low when an Alarm 1 or Alarm 2 match occurs and and the interrupt is enabled. The pin remains low until both alarm flags are cleared by the application.


\begin{DoxyCode}
1 \{c++\}
2 // Poll alarm 1 flag
3 if (rtc.getAlarmFlag(Alarm1)) \{
4     // Handle Alarm 1
5 
6     // Clear alarm 1 flag
7     rtc.clearAlarmFlag(Alarm1);
8 \}
9 
10 // Poll alarm 2 flag
11 if (rtc.getAlarmFlag(Alarm2)) \{
12     // Handle Alarm 2
13 
14     // Clear alarm 2 flag
15     rtc.clearAlarmFlag(Alarm2);
16 \}
\end{DoxyCode}


{\bfseries Alarm interrupt}

Note\+: Enabling interrupt will disable the {\ttfamily S\+QW} output signal.


\begin{DoxyCode}
1 \{c++\}
2 // Uno, Nano, Mini, other 328-based: pin D2 (INT0) or D3 (INT1)
3 #define INT\_PIN     2
4 
5 // Alarm interrupt flag must be volatile
6 static volatile bool alarmInterrupt = false;
7 
8 
9 static void alarmHandler()
10 \{
11     // Set global interrupt flag
12     alarmInterrupt = true;
13 \}
14 
15 void setup()
16 \{
17     ...
18 
19     // Attach to INT0 interrupt falling edge
20     pinMode(INT\_PIN, INPUT\_PULLUP);
21     attachInterrupt(digitalPinToInterrupt(INT\_PIN), alarmHandler, FALLING);
22 
23     // Enable Alarm 1 and 2 interrupts
24     rtc.alarmInterruptEnable(Alarm1, true);
25     rtc.alarmInterruptEnable(Alarm2, true);
26 \}
27 
28 void loop()
29 \{
30     // Check global alarm interrupt flag
31     if (alarmInterrupt) \{
32         if (rtc.getAlarmFlag(Alarm1)) \{
33             // Handle alarm 1
34 
35             // Clear alarm 1 interrupt
36             rtc.clearAlarmFlag(Alarm1);
37         \}
38 
39         if (rtc.getAlarmFlag(Alarm2)) \{
40             // Handle alarm 2
41 
42             // Clear alarm 2 interrupt
43             rtc.clearAlarmFlag(Alarm2);
44         \}
45     \}
46 \}
\end{DoxyCode}


{\bfseries 32k\+Hz clock out}

Enable or disable {\ttfamily 32k\+Hz} output pin.


\begin{DoxyCode}
1 \{c++\}
2 rtc.outputClockPinEnable(true);     // Enable 
3 rtc.outputClockPinEnable(false);    // Disable
\end{DoxyCode}


{\bfseries Square Wave Out (S\+QW)}

Note\+: Enabling {\ttfamily S\+QW} pin will disable the alarm {\ttfamily I\+NT} signal.


\begin{DoxyCode}
1 \{c++\}
2 rtc.setSquareWave(SquareWaveDisable);   // Disable
3 rtc.setSquareWave(SquareWave1Hz);       // 1Hz
4 rtc.setSquareWave(SquareWave1024Hz);    // 1024Hz
5 rtc.setSquareWave(SquareWave4096Hz);    // 4096Hz
6 rtc.setSquareWave(SquareWave8192Hz);    // 8192Hz
\end{DoxyCode}


\subsection*{Library dependencies}


\begin{DoxyItemize}
\item {\ttfamily Wire.\+h}
\item {\ttfamily Terminal.\+ino} requires {\ttfamily Erriez\+Serial\+Terminal} library.
\end{DoxyItemize}

\subsection*{Library installation}

Please refer to the \href{https://github.com/Erriez/ErriezArduinoLibrariesAndSketches/wiki}{\tt Wiki} page.

\subsection*{Other Arduino Libraries and Sketches from Erriez}


\begin{DoxyItemize}
\item \href{https://github.com/Erriez/ErriezArduinoLibrariesAndSketches}{\tt Erriez Libraries and Sketches} 
\end{DoxyItemize}